\documentclass[12pt]{article}

%% preamble: Keep it clean; only include those you need
\usepackage{amsmath}
\usepackage[margin = 1in]{geometry}
\usepackage{graphicx}
\usepackage{booktabs}
\usepackage{natbib}
\usepackage{enumitem}

% for space filling
\usepackage{lipsum}
% highlighting hyper links
\usepackage[colorlinks=true, citecolor=blue]{hyperref}


%% meta data

\title{Proposal: Bridging Gaps: Investigating COVID-19's Influence on Health Disparities in Connecticut}
\author{Delia Lin\\
  Department of Statistics\\
  University of Connecticut
}

\begin{document}
\maketitle


\paragraph{Introduction}
Social determinants of health (SDOH) are the conditions in which people are born, 
grow, live, work, and age, which significantly influence their overall health and 
well-being. These determinants include factors such as socioeconomic status, education, 
access to healthcare, and the physical environment. Understanding the interactions of 
these elements will be essential for addressing health disparities and developing more 
effective public health policies and interventions.

Current research focuses on how social determinants of health (SDoH) plays a  massive
impact on one's health; it is estimated that 80 percent of a population's health outcomes are 
dictated by SDoH \citep{HOOD2016129}. Often, SDoH, when referring to an individual, can result in racial 
disparities in care when looking at a population\citep{Monroe2023-uq}. It has been shown that major inefficiencies
in the health system are attributed to overlooked prevention opportunities and unequal access
to care.\citep{Allin2014-xn}

\paragraph{Specific Aims}
The question aims to assess the influence of COVID-19 on key social determinants of health 
within various counties and racial groups in Connecticut. While the influence of social factors 
on healthcare access is widely acknowledged, this research question was selected to address the 
pressing need for understanding how the pandemic amplified existing disparities across diverse 
communities. By investigating these factors, we can identify the 
specific ways in which different communities were affected, providing insights for 
targeted interventions, policy-making, and equitable healthcare strategies. 
Current research has shown that in the United States, predominantly black counties experience a 
COVID-19 infection rate that is threefold higher and a mortality rate that is six times higher 
compared to predominantly white counties \citep{Yancy2020-cz}. In addition, during the pandemic, mortality 
rates among historically marginalized minority communities have been 1.9 to 2.4 times higher 
compared to the general population.\citep{Badalov2022-wt}

Research question: What is the extent of COVID-19's impact on social determinants of health factors, including socioeconomic status, housing, transportation, and Medicare access, when comparing the counties and racial groups in Connecticut? 

Statistical Questions:
\begin{enumerate}
\item
What is the racial composition within each county?
\item
Is there a correlation between median income and racial identification between the counties?
\item
Is there a correlation between median income and education between the counties?
\item
Is there a difference between transportation accessibility between the counties in Connecticut?
\item
Is there a difference between primary care physician shortage percentages between counties?
\item
Was there an impact on the median household income for any specific racial groups or counties during the year 2020?
\item
Was there a shift in the proportion of the population within different poverty ratios in Connecticut during the year 2020?
\item
Was there a reduction in the percentage of housing units with heating in Connecticut during the year 2020?
\item
Was there a shift in the percentage of Connecticut's population covered by Medicaid or lacking insurance during the year 2020?
\end{enumerate}


\paragraph{Data}
Data was collected from The Agency for Healthcare Research and Quality (AHRQ).
The dataset comprises 64 variables spanning a period of 4 years (2017--2020) 
with observations across the 8 counties in Connecticut. These variables encompass 
a total of 2048 observations. The variables questions include housing, education level, 
income, access to electronic devices, transportation, ESL, and population racial characteristics. 
The dataset includes a range of calculated percentages, median values, and raw observations, 
providing a holistic view of various factors affecting the communities in these counties. 



\paragraph{Research Design and Methods}

In this study, descriptive statistics will be utilized to outline the racial composition, 
median incomes, education levels, transportation accessibility, and healthcare indicators 
across different counties and racial groups in Connecticut. Multivariable logistic regression 
will be used to estimate the adjusted odds ratio associations for different SDH factors on 
racial identification over the study period.\citep{Cova2023-yw} Additionally, chi-squared tests will be employed 
to assess significant shifts in population percentages between counties over the span of the 4 years.\citep{Lua2023-tp}


\paragraph{Discussion}
The most challenging part of this investigation would be working with the dataset which 
involves multiple variables and diverse demographics. Managing missing or incomplete data points,
and addressing potential biases in the dataset resulting from surveys are important. 

One limitation lies in the availability and quality of data. This dataset does not have any data 
from years after 2020 which may serve to limit potential external validity considerations. There may
also be variability in data collection methods and discrepancies in reporting standards leading to 
missing or incomplete data over the course of 4 years. Another limitation involves the scope of the 
study, focusing on specific counties in Connecticut may not fully capture nationwide disparities. 
Additionally, the research is limited to the factor parameters selected to investigate which  might 
not encompass all relevant social determinants affecting health outcomes.

In the face of unexpected events, I will prioritize transparency. If data limitations arise, efforts 
will be made to clearly document these constraints and their potential impact on the study's conclusions. 
If the impact of COVID is not well represented by the incomplete data in this dataset, I will also look 
into obtaining other existing datasets with observations collected during the study period and after 2020 
to ensure comprehensiveness. Open communication about challenges and potential limitations will be 
maintained to ensure the research's integrity and to provide a foundation for future studies in the field.

Following analysis, I expect to find a greater prevalence of social risk factors within minority populations, 
counties with lower median income will have a greater proportion of minority populations, and social risk 
factors display increasing trends in 2020. These expectations stem from recent reports that have shown that 
minority populations have been disproportionately impacted by the effects of the COVID-19 pandemic. The impact 
of this work lies in informing targeted interventions, policy-making, and resource allocation to alleviate 
current healthcare inequities remnant of the pandemic as well as inform future action. Even if the results 
of this investigation are not what I expected, any conclusions that can be made about socioeconomic factors 
between counties are important for future policies.

This research proposal delves into the impact of COVID-19 on social determinants of health, focusing on 
socioeconomic status, housing, transportation, and Medicare access among Connecticut counties and racial groups. 
Through this research, I hope to be able to enhance the understanding of the connections between social 
determinants and health outcomes specifically in Connecticut, fostering positive changes in healthcare 
systems and advocating for a more equitable society.


\bibliography{../manuscript/ref}
\bibliographystyle{chicago}

\end{document}